\documentclass[conference]{IEEEtran}
\IEEEoverridecommandlockouts
% The preceding line is only needed to identify funding in the first footnote. If that is unneeded, please comment it out.
\usepackage{cite}
\usepackage{amsmath,amssymb,amsfonts}
\usepackage{algorithmic}
\usepackage{graphicx}
\usepackage{textcomp}
\usepackage{xcolor}
\def\BibTeX{{\rm B\kern-.05em{\sc i\kern-.025em b}\kern-.08em
    T\kern-.1667em\lower.7ex\hbox{E}\kern-.125emX}}
\begin{document}

\title{Logistic Regression\\
%{\footnotesize \textsuperscript{*}Note: Sub-titles are not captured in Xplore and
%should not be used}
%\thanks{Identify applicable funding agency here. If none, delete this.}
}

\author{\IEEEauthorblockN{Pena Benafa}
\IEEEauthorblockA{\textit{Electronic Engineering} \\
\textit{University of Applied Science Hamm-Lippstadt}\\
Lippstadt, Germany \\
pena.benafa@stud.hshl.de}
}

\maketitle

\begin{abstract}

\end{abstract}

\begin{IEEEkeywords}
component, formatting, style, styling, insert
\end{IEEEkeywords}

\section{Introduction}


\subsection{Machine Learning Algorithms and Examples}


\subsection{Description of the Classification Problem}


\section{Definition of Logistic Regression}

\subsection{Dependent Variables}

\subsection{Independent Variables}


\section{Logistic Regression Model}

\subsection{Regression Equation}

\subsection{Regression Curve}
\subsection{Decision Boundary}
\subsection{Non-Linear Decision Boundary}

\section{Cost Function and Gradient Descent}

\section{Over fitting Problem}
\subsection{Under Fitting}
\subsection{Over Fitting}

\section{Regularized Logistic Regression}

\section{Conclusion}

\nocite{b1}

\nocite{b2}


\section*{Acknowledgment}


%\section*{References}
\bibliographystyle{IEEEtran}
\bibliography{reference}


%\begin{thebibliography}{00}





%\end{thebibliography}

\end{document}
